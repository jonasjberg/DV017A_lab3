\section{Uppgift 1}\label{sec:uppg01}

\subsection{Instruktioner}
\begin{verbatim}
1. Hur visar man att en referensvariabel inte refererar (pekar) på något
   objekt? Utgå från följande lilla kodsnutt:

   //Här refererar (pekar) kalle till ett nytt Person-objekt
   Person kalle = new Person ("Kalle Persson");
   //Hur skriver du här om du vill att referensen kalle *inte*
   //ska peka på något objekt ????????
\end{verbatim}


\subsection{Svar}
Om referensen \texttt{kalle} inte ska peka på något objekt så måste
referensvariabeln \texttt{kalle} sättas till att peka på \emph{ingenting},
eller \texttt{null}.  Referensvariabeln sätts till \texttt{null} genom att man
skriver:
\begin{verbatim}kalle = null;\end{verbatim}


%\subsection{Källkod}
%\javacode{src/main/Lab3Uppg01.java}
%\caption{Lab3Uppg01.java}
%\label{src:uppg01}


%\subsection{Skärmdump}
%\begin{figure}[htbp]
    %\centering
        %\includegraphics[width=\linewidth]{img/01.png}
    %\caption{Körning av koden till Uppgift~\ref{sec:uppg01}}
    %\label{fig:uppg01-screenshot}
%\end{figure}

