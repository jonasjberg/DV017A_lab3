\section{Uppgift 3}\label{sec:uppg03}

\subsection{Instruktioner}
Skriv en klass Cirkel som representerar en cirkel. I klassen ska följande
två instansvariabler ingå:

\begin{itemize}
\itemsep1pt\parskip0pt\parsep0pt
\item radie (int)
\item färg (String)
\end{itemize}

I parentes står lämplig datatyp.

I klassen ska finnas en metod för var och en av följande operationer:

\begin{itemize}
\itemsep1pt\parskip0pt\parsep0pt
\item ändra färgen på cirkeln
\item returnera cirkelns färg
\item ändra cirkelns radie
\item returnera cirkelns radie
\item beräkna och returnera cirkelns omkrets
\item beräkna och returnera cirkelns area
\end{itemize}


När man ska initiera ett cirkel-objekt ska man kunna välja på följande två
alternativ:

\begin{itemize}
\itemsep1pt\parskip0pt\parsep0pt
\item välja \emph{både} färg och radie på cirkeln
\item välja endast radien på cirkeln, men cirkelns färg blir alltid gul
\end{itemize}

Detta betyder att du behöver två stycken konstruktorer (som överlagrar
varandra) i din klass.

Slutligen skriv ett testprogram som testar så att klassens metoder fungerar
som de ska.  Skapa åtminstone två stycken cirkel-objekt, som använder sig av
olika konstruktorer. Alla decimaltals-utskrifter på skärmen ska avrundas
till \emph{en} decimal.

Tips: För att vid utskrift avrunda ett decimaltal använd klassen
DecimalFormat, finns exempel i boken hur du använder denna. Den ligger i
paketet java.text, du måste alltså importera denna klass i ditt testprogram.


\subsection{Källkod}
\subsubsection{Lab3Uppg03.java}
\javacode{src/Lab3Uppg03/Lab3Uppg03.java}
\caption{Lab3Uppg03.java}
\label{src:uppg03}

\subsubsection{Cirkel.java}
\javacode{src/Lab3Uppg03/Cirkel.java}
\caption{Cirkel.java}
\label{src:cirkel}


\subsection{Skärmdump}
Se Figur~\ref{fig:uppg03-screenshot} för skärmdump på körning av koden i
Sektion~\ref{src:uppg03}.

%\begin{figure}[h]
\begin{figure}[htbp]
    \centering
        \includegraphics[width=\linewidth]{img/03.png}
        \caption{Körning av koden i Sektion~\ref{src:uppg03} tillhörandes Uppgift~\ref{sec:uppg03}}
    \label{fig:uppg03-screenshot}
\end{figure}

