\section{Uppgift 2}\label{sec:uppg02}

\subsection{Instruktioner}
\begin{verbatim}
Skriv en klass FlygPlan som representerar ett flygplan. Klassens
instansvariabler är:

* höjd (int)
* flygriktning (int)
* hastighet (int)
* modellbeteckning (String)

Lämplig datatyp står inom parentes. När det gäller instansvariabeln
flygriktning så ska den ha värdet 0 om planet står stilla, värdet 1 vid
nordlig riktning, 2 ostlig, 3 sydlig och 4 västlig.

Metoder ska finnas för följande operationer:

- ändra planets höjd
- returnera planets höjd
- ändra planets flygriktning
- returnera planets flygriktning
- ändra planets hastighet
- returnera planets hastighet
- ändra planets modellbeteckning
- returnera planets modellbeteckning
- skriv ut alla data om flygplanet, metoden ska ha returtypen void

Glöm ej konstruktorn.

Skriv sedan ett testprogram som testar klassens metoder. Skapa minst 2
stycken FlygPlans-objekt och testa metoderna på dessa objekt.
\end{verbatim}


\subsection{Kommentar}
Jag har valt att använda datatypen \texttt{enum} för att lösa uppgiften.  I
instruktionerna efterfrågas numeriska värden för flygriktningen och min lösning
uppfyller det kravet genom att metoden \texttt{getordinal()} från typen
\texttt{enum} används för att hämta de implicita värden som tillskrivs de olika
flygriktningarna i \texttt{Flygriktning} efter den ordningen de deklareras.
Min lösning för att \emph{sätta} värdet hos en variabel av typen
\texttt{Flygriktning} med en \texttt{int} är bristfällig, bättre vore att
basera lösningen på de mer avancerade funktionerna som finns ``inbyggda'' i
typen \texttt{enum}.

Användandet av \texttt{enum} baseras på information från
\mbox{The Java Tutorials -- Enum Types}.
\footnote{\url{https://docs.oracle.com/javase/tutorial/java/javaOO/enum.html}}


\subsection{Källkod}
\subsubsection{Lab3Uppg02.java}
\javacode{src/Lab3Uppg02/Lab3Uppg02.java}
\caption{Lab3Uppg02.java}
\label{src:uppg02}

\subsubsection{Flygriktning.java}
\javacode{src/Lab3Uppg02/Flygriktning.java}
\caption{Flygriktning.java}
\label{src:flygriktning}

\subsubsection{FlygPlan.java}
\javacode{src/Lab3Uppg02/FlygPlan.java}
\caption{FlygPlan.java}
\label{src:flygplan}


% Screenshots med Bash, terminalfönsterstorlek 90x40
\subsection{Skärmdump}
\begin{figure}[htbp]
    \centering
        \includegraphics[width=\linewidth]{img/02.png}
    \caption{Körning av koden till Uppgift~\ref{sec:uppg02}}
    \label{fig:uppg02-screenshot}
\end{figure}

